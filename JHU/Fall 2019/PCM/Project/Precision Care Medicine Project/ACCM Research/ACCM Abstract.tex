\documentclass{article}[10pt,letterpaper]
\usepackage{cvpr}
\usepackage{times}
\usepackage{epsfig}
\usepackage{graphicx}
\usepackage{amsmath}
\usepackage{amssymb}

% Include other packages here, before hyperref.

% If you comment hyperref and then uncomment it, you should delete
% egpaper.aux before re-running latex.  (Or just hit 'q' on the first latex
% run, let it finish, and you should be clear).
%\usepackage[breaklinks=true,bookmarks=false]{hyperref}

\cvprfinalcopy % *** Uncomment this line for the final submission
 

\title{Advanced Risk Stratification and Prediction of Venous Thromboembolism in Critically Ill Patients}

\author{
Ryan Brody\thanks{Department of Biomedical Engineering, Johns Hopkins Whiting School of Engineering}
\and Jinrui Liu\footnotemark[1]
\and Akilan Meiyappan\footnotemark[1] 
\and Bronte Wen\footnotemark[1]
\and Elizabeth Wu\footnotemark[1]
\and Dr. Raimond Winslow\footnotemark[1] 
\and Dr. Joseph Greenstein\footnotemark[1]
\and Dr. Sachidanand Hebbar \thanks{Department of Anesthesiology, Johns Hopkins Hospital}
\and Dr. Nauder Faraday\footnotemark[2]
\and Dr. Adam Sapirstein\footnotemark[2]
}

%{Department of\^1 Anesthesiology, Johns Hopkins Hospital; \^2Department of Biomedical Engineering, Johns Hopkins Whiting School of Engineering}
\begin{document}
\maketitle

\section{Introduction:}
Venous thromboembolism (VTE) is a preventable disorder that includes both pulmonary embolism (PE) and deep venous thrombosis (DVT), and is estimated to be responsible for more than 100,000 deaths per year\cite{SG}. VTE is particularly problematic in the intensive care setting. In spite of continued improvements to prophylaxis and treatment regimens, studies have shown an incidence of up to 10\% in medical-surgical ICUs. Diagnosis of VTE in the critically ill remains challenging, as clinical scoring systems such as the Modified Wells’ criteria have poor sensitivity and rely on patient reported symptoms and low frequency physiologic monitoring. Need to balance risks of treatment. Data from ICU patients in real time physiologic trends, underlying disease states, patient characteristics and therapies, no current risk stratification model utilizes all of this data in a unified manner, particularly the real time information. As such, the purpose of this project is to develop a real-time predictive model that will determine if a patient is at risk for developing VTE, in a clinical environment.
\section{Methods}
In order to properly develop the model, the project will be divided into three objectives. The first objective is to pre-process the dataset and identify features of predictive significance. Pre-processing the data details selecting patients that developed VTE after being admitted to the ICU, while features of predictive significance will be identified through Principal Component Analysis, studying features used in previous models such as Caprini and receiving professional input from clinicians. The second objective is to design the risk assessment model for VTE. This will be accomplished by testing several different conventional regression and machine learning algorithms to determine which algorithm can correlate relevant features to a positive patient ID for VTE. The final objective is to design and validate a real-time predictive model for VTE. The previous risk assessment model will be expanded such that it will be able to provide real time analysis on the likelihood of a patient developing VTE. 

\section{Results}
The project is currently in its infancy stage, as such the only reportable data correlates to positive patients for VTE. Currently out of 10,182 patients, 363 are positive for VTE with 190 diagnosed with DVT, 132 diagnosed with PE and 41 diagnosed with both DVT and PE. 
\section{Conclusion}
Based on the evidence provided, it is apparent that there is not only a need for risk assessment model for VTE, but also sufficient enough data to produce such model. Furthermore, producing such model will save countless lives for years to come. 

\bibliographystyle{plain}
\bibliography{ref.bib}

\end{document}